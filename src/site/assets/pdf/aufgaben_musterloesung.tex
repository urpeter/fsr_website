\documentclass[a4paper]{article}
\usepackage[utf8]{inputenc}
\usepackage[T1]{fontenc}
\usepackage{lmodern}
\usepackage[ngerman]{babel}
\usepackage{amsmath}
\usepackage{enumitem}
\usepackage{wasysym}
\usepackage{graphicx}
\usepackage{float}
\usepackage{geometry}
\usepackage{csquotes}
\usepackage{booktabs}

\geometry{
       paperwidth=210mm,
       paperheight=290mm,
       left=20mm,
       right=20mm,
       top=20mm,
       bottom=20mm}

\begin{document}
\section{Ein bisschen Text}
\subsection{Erster Teil der Erzählung}
Er hörte leise Schritte hinter sich. Das bedeutete nichts Gutes. Wer würde ihm schon folgen, spät in der Nacht und dazu noch in dieser engen Gasse mitten im übel beleumundeten Hafenviertel? Gerade jetzt, wo er das Ding seines Lebens gedreht hatte und mit der Beute verschwinden wollte!

Hatte einer seiner zahllosen Kollegen dieselbe Idee gehabt, ihn beobachtet und abgewartet, um ihn nun um die Früchte seiner Arbeit zu erleichtern? Oder gehörten die Schritte hinter ihm zu einem der unzähligen Gesetzeshüter dieser Stadt, und die stählerne Acht um seine Handgelenke würde gleich zuschnappen? Er konnte die Aufforderung stehen zu bleiben schon hören.

Gehetzt sah er sich um. Plötzlich erblickte er den schmalen Durchgang. \emph{Blitzartig} drehte er sich nach rechts und verschwand zwischen den beiden Gebäuden. Beinahe wäre er dabei über den umgestürzten Mülleimer gefallen, der mitten im Weg lag. Er versuchte, sich in der Dunkelheit seinen Weg zu ertasten und erstarrte: \textbf{Anscheinend gab es keinen anderen Ausweg aus diesem kleinen Hof als den Durchgang, durch den er gekommen war.}

\subsection{Zweiter Teil der Erzählung}
Die Schritte wurden lauter und lauter, er sah eine dunkle Gestalt um die Ecke biegen. Fieberhaft irrten seine Augen durch die nächtliche Dunkelheit und suchten einen Ausweg. War jetzt wirklich alles vorbei, waren alle Mühe und alle Vorbereitungen umsonst?

Er presste sich ganz eng an die Wand hinter ihm und hoffte, der Verfolger würde ihn übersehen, als plötzlich neben ihm mit \texttt{kaum wahrnehmbarem Quietschen} eine Tür im nächtlichen Wind hin und her schwang. Könnte dieses der flehentlich herbeigesehnte Ausweg aus seinem Dilemma sein? Langsam bewegte er sich auf die offene Tür zu, immer dicht an die Mauer gepresst. Würde diese Tür seine \textbf{Rettung} werden?

\section{Ein Gedicht}
Über allen Gipfeln\\
Ist Ruh,\\
In allen Wipfeln\\
Spürest du\\
Kaum einen Hauch;\\
Die Vögelein schweigen im Walde.\\
Warte nur, balde\\
Ruhest du auch.

\section{Teilgebiete der Linguistik}
\begin{enumerate}[label=\Roman*]
\item Phonetik und Phonologie
\item Morphologie und Syntax
\item Semantik
\item Pragmatik und Diskurs
\item Psycholinguistik
\end{enumerate}

\section{Beliebtes Mensa-Essen}
\begin{itemize}[label=\smiley]
\item Schnitzel
\item Pommes
\item Teigwaren
\item Seelachsfilet
\item Pommes
\item Schnitzel
\end{itemize}

\section{Eine Tabelle}
\begin{table}[H]
\centering
\begin{tabular}{llc}
\toprule
\textbf{Name} & \textbf{Vorlesung} & \textbf{Note}\\ \midrule
Max Mustermann & Computerlinguistik & 2{,}0\\
		~		& ASW & 1{,}3\\
		~		& Syntax & 2{,}7\\
Sebastian Shrdlu & Programmierung 1 & 1{,}0\\
		~		& Mathe 1 & 1{,}7	\\	
		~		& Psycholinguistik & 3{,}3\\ \bottomrule

\end{tabular}
\caption{Notenspiegel}
\end{table}

\section{Ein Bild}
\begin{figure}[H]
\centering \includegraphics[scale=0.2]{funnycat.jpg}
\caption{\textit{Felis catus}}
\end{figure}

\section{Etwas Mathe}
Die Gleichung $a^n + b^n = c^n$ hat keine ganzzahligen Lösungen für $ n > 2$.

\begin{equation*}
H(X) = \sum_{x \in X} p(x) \log_2 \frac{1}{p(x)}
\end{equation*}

\begin{align*}
  (a + b) \cdot (a - b)
  &= a^2 - ab + ab - b^2 \\
  &= a^2 - b^2
\end{align*}

\section{Eine eingebundene Datei}
\input{ipsum.tex}

\end{document}